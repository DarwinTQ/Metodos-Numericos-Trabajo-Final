\documentclass{article}
\usepackage{listings}
\usepackage{xcolor}

% Configuración del estilo para mostrar código
\lstset{
    language=Python,          % El lenguaje que estás utilizando (puedes cambiarlo)
    basicstyle=\ttfamily\small, % Estilo de la fuente del código
                % Colocar números de línea a la izquierda
    numberstyle=\tiny\color{gray}, % Color y tamaño de los números de línea
    stepnumber=1,             % Colocar un número en cada línea
    numbersep=10pt,           % Separación entre el código y los números de línea
    backgroundcolor=\color{lightgray!10}, % Color de fondo del código
    showspaces=false,         % No mostrar espacios como caracteres visibles
    showstringspaces=false,   % No mostrar espacios en las cadenas de texto
    showtabs=false,           % No mostrar tabulaciones como caracteres visibles
    frame=single,             % Marco alrededor del código
    rulecolor=\color{black},  % Color del marco
    tabsize=4,                % Tamaño de las tabulaciones
    captionpos=b,             % Posición de la leyenda (b para abajo)
    breaklines=true,          % Romper las líneas largas
    breakatwhitespace=true,   % Romper solo en espacios en blanco si es posible
    keywordstyle=\color{blue},    % Color de las palabras clave
    commentstyle=\color{green!50!black}, % Color de los comentarios
    stringstyle=\color{red},       % Color de las cadenas de texto
}

\begin{document}

\title{Código en \LaTeX}
\author{Tu Nombre}
\date{\today}
\maketitle

\section{Ejemplo de Código}

Aquí tienes un ejemplo de cómo insertar código en \LaTeX.

\begin{lstlisting}
   
    a = input("Ingrese el valor de a= ");
    b = input("Ingrese el valor de b= ");
    tol = input("Ingrese el valor de tol= ");
    f = input("Ingrese la funcion f=", 's');
    f = inline(f);  
    k = 1;          
    m = (a + b) / 2; 
    
    fprintf('k\tak\t\tbk\t\tmk\t\terror\n');
    
    fa = f(a);
    fm = f(m);
    
    while (b - a) / 2 > tol
        fprintf('%d\t%f\t%f\t%f\t%f\n', k, a, b, m, (b - a) / 2);
        
        if fa * fm < 0  
            b = m;      
        else
            a = m;      
            fa = fm;    
        end
        
        m = (a + b) / 2;
        fm = f(m); 
        k = k + 1;  
    end
    fprintf('La raiz aproximada es m = %f\n', m);
\end{lstlisting}

\end{document}
